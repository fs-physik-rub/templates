\documentclass[a4paper, 12pt]{article}
\usepackage[T1]{fontenc}
\usepackage[utf8]{inputenc}
\usepackage[ngerman]{babel}
\usepackage{a4wide}
\usepackage{color}
\usepackage[dvipsnames]{xcolor}
\usepackage{hyperref}
\usepackage{ulem}
\usepackage{enumitem}

\newcommand{\changefont}[3]{\fontfamily{#1}\fontseries{#2}\fontshape{#3}\selectfont}

\newcommand{\streichen}[1]{\textcolor{red}{\sout{}}}
\newcommand{\neu}[1]{#1}
\newcommand{\fabian}[1]{#1}
\newcommand{\streichenfabian}[1]{\textcolor{blue}{\sout{}}}

\title{\textsc{Bestimmungen zum Haushalt 2020 der Fachschaft für Physik und Astronomie der Ruhr-Universität Bochum}}
\date{27.Februar 2020}

\begin{document}
\renewcommand{\labelenumi}{§ \arabic{enumi}}
\renewcommand{\labelenumii}{(\arabic{enumii})}
\renewcommand{\labelenumiii}{\alph{enumiii})}

\newcommand{\cen}[1]{\vspace{0.25cm}\textsc{#1}
						\vspace{0.25cm}}
\newcommand{\cc}[1]{\vspace{0.25cm}\begin{center} \Large\bf \textsc{#1} \end{center}}

\changefont{ppl}{m}{n}

\maketitle
\thispagestyle{empty}
	
	
	\begin{enumerate}[leftmargin=0cm]
	
	\item \cen{Geltungsbereich}	
	
	Diese Bestimmungen gelten für den Haushaltsplan 2020 der Fachschaft für Physik und Astronomie der Ruhr-Universität Bochum, welcher für das Haushaltsjahr vom 1. März 2020 bis zum 28. Februar 2021 gilt.

	\item \cen{Überplanmäßige Einnahmen}
	
	Der Rat kann im Falle überplanmäßiger Einnahmen, beispielsweise durch Überweisungsrückläufer, Rückzahlungen oder Mehreinnahmen, den Ausgabenansatz eines Titels um diese Summe überschreiten.

	\item \cen{Flexibilisierte Ausgaben}
	
	\begin{enumerate}[leftmargin=0cm]
		\item Die Haushaltstitel 2001, 2003 und 2004 sind untereinander gegenseitig deckungsfähig.
		\item Die Haushaltstitel 3001, 3002 und 3003 sind untereinander gegenseitig deckungsfähig.
	\end{enumerate}

	\item \cen{Voraussetzung für Auszahlungen}

	Voraussetzung für die Auszahlung von Geldern ist eine Freigabe der Gelder durch den Rat für die jeweilige Angelegenheit.
	 
	
	\end{enumerate}

\end{document}