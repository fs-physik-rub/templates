\documentclass[12pt,a4paper]{g-brief2}
\usepackage[utf8]{inputenc}
\usepackage[ngerman]{babel}
\usepackage[T1]{fontenc}
%\usepackage[oldstyle]{libertine}
% Wenn es Probleme mit libertine gibt, die vorherige Zeile auskommentieren
% und die nächsten drei Zeilen verwenden:
\usepackage{mathpazo}
\usepackage{helvet}
\usepackage{courier}
\usepackage[tracking=true]{microtype}
\usepackage{color}
\usepackage{graphicx}
\usepackage{calc}
\usepackage{fancyhdr}
\usepackage{lipsum}

\DeclareMicrotypeSet*[tracking]{my}{ font = */*/*/sc/* }\SetTracking{ encoding = *, shape = sc }{ 45 }

\rhead[]{}
\chead[]{}
\lhead[]{}
\rfoot[]{}
\cfoot[Seite~\thepage{} von~\pageref{end}]{Seite~\thepage{} von~\pageref{end}}
\rfoot[]{}
\pagestyle{fancy}
\setlength{\parskip}{\medskipamount}
\setlength{\parindent}{0pt}
\newcommand{\fslogo}{\parbox{\textwidth}{\centering \includegraphics[height=0.585in]{Logo.jpg}}}

\makeatletter

\fenstermarken
\faltmarken
\trennlinien
\unserzeichen

\makeatother

\usepackage{babel}

\begin{document}
\thispagestyle{empty}
\Name{\fslogo}

\NameZeileA{Ruhr-Universität Bochum}
\NameZeileB{Fachschaft Physik und Astronomie}
\NameZeileC{}
\NameZeileD{\textbf{Raum:}}
\NameZeileE{NB 02/174}
\NameZeileF{}
\NameZeileG{}

\AdressZeileA{Ruhr-Universität Bochum}
\AdressZeileB{Fachschaft Physik und Astronomie}
\AdressZeileC{Fach NC 116}
\AdressZeileD{Universitätsstraße 150}
\AdressZeileE{44801 Bochum}
\AdressZeileF{DEUTSCHLAND}

\TelefonZeileA{+49 (0)234 32-28027}
\TelefonZeileB{}
\TelefonZeileC{}
\TelefonZeileD{}
\TelefonZeileE{}
\TelefonZeileF{}

\InternetZeileA{https://fachschaft.physik.rub.de/}
\InternetZeileB{}
\InternetZeileC{\textbf{Fachschaft:}}
\InternetZeileD{fachschaft@physik.rub.de}
\InternetZeileE{\textbf{Fachschaftsrat:}}
\InternetZeileF{fs-rat@physik.rub.de}

\BankZeileA{}
\BankZeileB{}
\BankZeileC{}
\BankZeileD{}
\BankZeileE{}
\BankZeileF{}

\Postvermerk{}

\RetourAdresse{RUB $\cdot$ FS Physik $\cdot$ NC 116 $\cdot$ Universitätsstr. 150 $\cdot$ D-44801 Bochum}
\Adresse{Herrn\\
  Max Mustermann}

%\MeinZeichen{Az der Fachschaft}
\MeinZeichen{}
%\IhrZeichen{Az der Fakultät}
\IhrZeichen{}
%\IhrSchreiben{1.\,April~2018}
\IhrSchreiben{}
\Datum{\today}
\Betreff{Bestätigung über Fachschaftsarbeit}
\Anrede{Sehr geehrte Damen und Herren,}
\Gruss{Mit freundlichen Grüßen,}{0.5cm}
\Unterschrift{\parbox{\textwidth}{
    \parbox{65mm}{\noindent\rule{65mm}{1pt}\\
      (Name des Sprechers)\\
      Sprecher des Rates}\hfill\parbox{70mm}{\noindent\rule{65mm}{1pt}\\
      (Name des stv. Sprechers)\\
      Stellvertretender Sprecher des Rates}}}
\Anlagen{}
\Verteiler{\label{end}}

\begin{g-brief}
  hiermit bestätigen wir, dass Herr Max Mustermann vom~Tag.\,Monat~Jahr
  bis zum~Tag.\,Monat Jahr im Rat der Fachschaft für Physik und Astronomie der
  Ruhr-Universität Bochum als gewähltes Mitglied tätig war.

  Herr Mustermann wurde durch den Rat bzw. die Fachschaft mit der folgenden Ämtern betraut:
  \begin{itemize}
  \item Sprecher (von - bis)
  \item Finanzreferent (von - bis)
  \item etc.
  \end{itemize}

  Herr Mustermann hat in folgenden Bereichen einen leitenden bzw. hervorzuhebenden Beitrag zur Arbeit der Fachschaft geleistet
  %Auf wichtiges beschränken/ mit dem Betreffenden absprechen
  \begin{itemize}
  	\item Planung und Durchführung der Erstsemestereinführung
  	\item Erstellung und Durchführung Umfrage unter den Physikstudierenden
  	\item Erstellung einer Satzung und einer Geschäftsordnung
  	\item Einrichtung eines Physik-Helpdesks
  	\item etc.
  \end{itemize}
  
  Herr Mustermann wurde durch den Rat in die folgenden Gremien der Fakultät entsandt:
  % Auf wesentliches beschränken/ mit dem Betreffenden absprechen
  \begin{itemize}
  \item Qualitätsverbesserungsausschuss (von - bis)
  \item Prüfungsausschuss (von - bis)
  \item Finanzausschuss (von - bis)
  \item etc.
  \end{itemize}

	Zudem ist Herr Mustermann für die Liste der Fachschaft Physik \& Astronomie bei der Fakultätsratswahl angetreten und wurde zum studentischen Vertreter im Fakultätsrat gewählt.
	%Hier kann man auf die Fakultät zur Erstellung eines Nachweises verweisen.
  
  Wir danken Herrn Mustermann für seine Tätigkeit in der Fachschaft.
  %Optional: Ein paar besonders herzliche Worte bei herausragenden Tätigkeiten.
\end{g-brief}
\end{document}
